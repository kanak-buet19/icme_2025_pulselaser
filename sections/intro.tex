\section{Introduction}
\usetikzlibrary{positioning, shapes.geometric, calc}
\usetikzlibrary{positioning, shapes.geometric, arrows.meta}
\begin{frame}{Introduction}

    \begin{block}{Laser Welding}
        One of the most preferred welding techniques due to excellent precision and high energy density -- produces excellent quality joints.
    \end{block}

    \vspace{0.5cm}

    \begin{center}
    \begin{tikzpicture}[
        scale=0.8, transform shape, 
        node distance=1cm and 1.5cm, 
        boxstyle/.style={
            rectangle, 
            draw=none, 
            fill=structure.fg!90!black, 
            text=white, 
            align=center, 
            rounded corners=3pt, 
            minimum height=1.2cm, 
            font=\small
        },
        line/.style={
            draw, 
            ->, 
            >=stealth,  
            thick, 
            color=structure.fg!90!black
        }
    ]

        \node[boxstyle, minimum width=3cm, minimum height=1.5cm] (root) {Laser Welding};

        \node[boxstyle, right=1.5cm of root, yshift=1.0cm, minimum width=3.5cm] (cw) {Continuous Wave};
        \node[boxstyle, right=1.5cm of root, yshift=-1.0cm, minimum width=3.5cm] (pw) {Pulsed Wave};

        \node[boxstyle, right=1cm of cw, text width=4.5cm] (cw_desc) {Laser interacts \\ with metal \\ continuously};
        \node[boxstyle, right=1cm of pw, text width=4.5cm] (pw_desc) {Laser applied in a \\ short intermittent \\ bursts};

        \draw[line] (root.east) -- (cw.west);
        \draw[line] (root.east) -- (pw.west);
        \draw[line] (cw.east) -- (cw_desc.west);
        \draw[line] (pw.east) -- (pw_desc.west);

    \end{tikzpicture}
    \end{center}

\end{frame}



\begin{frame}{Our work}
    \begin{block}{Why Pulsed Laser welding}
        It enables finer control of heat input and melt-pool dynamics – thus controlling the cooling rate and grain growth.
    \end{block}

    \begin{block}{Why Nickel Based Alloy}
        They have excellent properties like low density yet high toughness, bio-compatibility, and excellent resistance to creep, corrosion, and high temperatures
    \end{block}

    % A stand-out box using tcolorbox
    % colback = background color (light teal)
    % colframe = border color (dark teal)
    \begin{tcolorbox}[
        colback=blue!10!white, 
        colframe=blue!80!black, 
        title=\textbf{Our Work}, 
        fonttitle=\large,
        arc=2mm,             
        boxrule=0.5mm,       
        left=6pt, right=6pt, top=6pt, bottom=6pt
    ]
        This work utilizes a \textbf{Cellular Automata (CA)} based microstructural simulation to establish \textit{predictive process-structure relationships}, directly informing the optimization of weld quality in the alloy \textbf{Inconel 625}.
    \end{tcolorbox}
    
\end{frame}